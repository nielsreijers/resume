%%%%%%%%%%%%%%%%%%%%%%%%%%%%%%%%%
%%% CV Jakša Tomović 06/2019 %%%%%%
%%%%%%%%%%%%%%%%%%%%%%%%%%%%%%%%%

\PassOptionsToPackage{dvipsnames}{xcolor}
\documentclass[10pt,a4paper]{../altacv}

\input{../glyphtounicode}
\pdfgentounicode=1

%Layout
\geometry{left=1cm,right=9cm,marginparwidth=6.8cm,marginparsep=1.2cm,top=1.25cm,bottom=1.25cm,footskip=2\baselineskip}

%Packages
\usepackage[utf8]{inputenc}
\usepackage[T1]{fontenc}
\usepackage[default]{lato}
\usepackage{hyperref}

%Colors

% old
%\definecolor{accent}{HTML}{000e17}
%\definecolor{heading}{HTML}{000e17}
%\definecolor{emphasis}{HTML}{696969}
%\definecolor{body}{HTML}{01415f}

\definecolor{accent}{HTML}{000e17}
\definecolor{heading}{HTML}{0077b5}
\definecolor{emphasis}{HTML}{696969}
\definecolor{body}{HTML}{000e17}

\colorlet{heading}{heading}
\colorlet{accent}{accent}
\colorlet{emphasis}{emphasis}
\colorlet{body}{body}

\renewcommand{\itemmarker}{{\small\textbullet}}
\renewcommand{\ratingmarker}{\faCircle}

\begin{document}
\name{Niels Reijers}
\tagline{Lead / Senior Software Developer}
%\photo{2.5cm}{BrunoAlves}
\personalinfo{
    \location{New Taipei City, Taiwan}
    \printinfo{\faBirthdayCake}{5th Feb 1977}
    \email{nielsreijers@gmail.com}
    \phone{+886 975 140 428}
%    \mailaddress{SR Njemačke 6, 10000}
    
    
    \github{\href{https://www.github.com/nielsreijers}{www.github.com/nielsreijers}}
    \hfill
    \linkedin{\href{https://www.linkedin.com/in/niels-reijers-4609602}{www.linkedin.com/in/niels-reijers-4609602}}
    \hfill
    \homepage{\color{heading}\href{https://www.nielsreijers.com/cv-single-page-version.pdf}{www.nielsreijers.com/cv-single-page-version.pdf}}
}

\begin{fullwidth}
\makecvheader

I have over 25 years’ experience in IT, in both commercial and research environments. I’ve designed and implemented many systems from the ground up, and worked as lead developer on large projects. I hold a PhD in computer science from National Taiwan University and published my work in my field’s top conference. Recently I'm mostly doing commercial work, but I’m always keeping an eye on new research developments.

\medskip

I have strong analytical skills and experience with a wide variety of technologies. One recurring theme throughout my career has been performance analysis. I’ve improved performance in tiny embedded virtual machines, large scale databases, and complex enterprise infrastructures slowed down by unexpected interactions. As hobby projects, I recently started studying category theory and Haskell, and am getting my feet wet in machine learning.

\medskip

Besides these technical aspects, I enjoy thinking about the softer side of software development. Are we solving the right problems? Which new technologies make us more productive, and which just swamp us with unnecessary choices? Is technology changing faster than ever, or is it slowing down? And what does that mean for how we develop software?

\end{fullwidth}

\cvsection[page1sidebar]{Experience}

\cvevent{Lead / Senior Software Developer}{DSW Health Insurance}{Apr 2004 -- Ongoing}{Schiedam, The Netherlands}

Apr 2005 - Jan 2007: Software developer responsible for maintaining the pharmaceutical claims processing software.

\medskip

Feb 2007 - Nov 2009: Oversaw the development of the new claims processing system as the lead developer.
\begin{itemize}
	\item\small{Delivered on time, it has been running reliably for 13 years and proven to be very maintainable, despite being the company's most complex system.}
	\item\small{It produced several spin-off products that became a company-wide standard for other teams, leading to improved efficiency and standardization.}
\end{itemize}

\medskip

After moving Taiwan to study Chinese and pursue a PhD, I was granted the opportunity to come back to work as a senior developer for a few months per year. This privilege was not typically offered to other employees, and I am grateful for this flexibility.


\begin{itemize}
	\item\small{Worked on and built many different systems, often with a focus on performance tuning.}
	\item\small{Initiated two projects that had a particularly large impact:
	\begin{itemize}
		\item[-]\small{The development of a system for company-wide monitoring of service calls. This gave valuable real-time insight in the behavior and performance of their systems, which they did not have before.}
		\item[-]\small{I identified a problem with the growing complexity of the deployment system. On my suggestion this was changed to an XML-based ‘convention over configuration' solution. This improved maintainability and reduced the burden for developers, who no longer have to write deployments scripts.}
	\end{itemize}
	}
	\item\small{Rebuilt a database conversion that an external supplier had been working on for months and failed to get working. My implementation took 1 week to build and ran in minutes rather than hours. This allowed the project to go live and convert our customer's data in an acceptable time frame.}
\end{itemize}

\medskip

\cvtag{C\#}
\cvtag{MS SQL Server}
\cvtag{PowerShell}
\cvtag{Python}
\cvtag{Jupyter}



\medskip\medskip\cvevent{Postdoctoral Researcher}{Academica Sinica, Taiwan Information Security Center}{Mar 2020 -- May 2021}{Taipei, Taiwan}

I worked with professor Yuh-Jye Lee on a project on disinformation.

My focus was on using social connections to correct disinformation.

\medskip

\cvtag{Golang}
\cvtag{HTML}
\cvtag{CSS}
\cvtag{JavaScript}
\cvtag{Chrome Extensions}


\newpage
{\marginpar{\vspace*{\dimexpr1pt-\baselineskip}\raggedright
%%%%%%%%%%%%%%%%%%%%%%%%%%%%%%% Languages %%%%%%%%%%%%%%%%%%%%%%%%%%%%%%%


\cvsection{Languages}
\cvskill{Dutch}{5}
\cvskill{English}{5}
\cvskill{Mandarin Chinese}{2}

%%%%%%%%%%%%%%%%%%%%%%%%%%%%%%% Others %%%%%%%%%%%%%%%%%%%%%%%%%%%%%%%

\cvsection{Interests}

\begin{itemize}
\item\small{Cycling}
\item\small{5k runner}
\item\small{Beginning Vipassana meditator}
\item\small{Volunteer at the Rotterdam International Film Festival, before moving to Taipei}
\item\small{Prefers CDs over vinyl or streaming}
\item\small{Led the winning company team in the 2013 Delft University of Technology programming contest}
\end{itemize}

%}}
\cvevent{Postdoctoral Researcher}{National Taiwan University, Wireless Networking and Embedded Systems Lab}{Feb 2019 -- Aug 2019}{Taipei, Taiwan}


I worked with professor Hsueh Chih-Wen on block chain research.

\medskip

\cvtag{C}
\cvtag{Block chain}



\medskip\medskip\cvevent{Doctoral Candidate}{Intel-NTU Connected Context Computing Center}{Sep 2011 -- Apr 2018}{Taipei, Taiwan}

Developed CapeVM:
\begin{itemize}
    \item\small{A Java virtual machine for resource-constrained Internet-of-Things devices such as the Atmel ATMEGA128.}
    \item\small{CapeVM uses on-device ahead-of-time compilation to native code to drastically improve performance compared to existing VMs in this class, which are one to two orders of magnitude slower than optimized C.}
\end{itemize}

My work improved the state of the art by:
\begin{itemize}
    \item\small{Improving performance to close to optimized C and thus reducing energy consumption, while maintaining platform independence.}
    \item\small{Providing a safe execution environment to protect devices from buggy or malicious code, at a cost comparable to existing native code solutions.}
\end{itemize}

The results were published in the field's top conference, SenSys 2018.

\medskip

\cvtag{C}
\cvtag{Embedded systems}
\cvtag{AVR assembly}
\cvtag{Java}
\cvtag{JVM bytecode}



\medskip\medskip\cvevent{Doctoral Candidate}{Delft University of Technology, Parallel and Distributed Systems Groep}{Sep 2002 -- Mar 2005}{Delft, The Netherlands}

Worked on various aspects of wireless sensor networks.

\begin{itemize}
	\item\small A quantitative comparison of localization algorithms, which was picked up by Elsevier and turned into a book chapter.
	\item\small Developed an efficient algorithm for code distribution.
	\item\small Real-world radio measurements which showed the behavior at the link layer to be much more complex than the highly simplified models often used in simulations.
\end{itemize}

\medskip

I supervised the lab exercises accompanying the Compiler Construction course.

\medskip

\cvtag{C}
\cvtag{MSP430 assembly}
\cvtag{Wireless Sensor Networks}
\cvtag{YACC}



\medskip\medskip\cvevent{Research Assistant}{Trinity College Dublin, Distributed Systems Group}{May 2001 -- Dec 2001}{Dublin, Ireland}

\begin{itemize}
\item\small{Research for my master thesis.}
\item\small{Work done in the "Anois" and "Cortex" European research projects on augmented reality games.}
\begin{itemize}
\item[-]\small{Worked on improving GPS localization.}
\item[-]\small{Designed a group communication protocol that allowed the game to continue when the network is partitioned, and recover when the connection is re-established.}
\end{itemize}
\end{itemize}

\cvtag{C++}
\cvtag{OpenGL}
\cvtag{GPS}




\medskip\medskip\cvevent{Software Developer}{Millidian}{Feb 2000 -- Apr 2001}{Rotterdam, The Netherlands}

\begin{itemize}
\item\small{First developer of internet startup Millidian}
\item\small{Involved in every aspect of launching our website:}
\begin{itemize}
\item[-]\small{Developing both the front end and back end}
\item[-]\small{Building and optimising the database}
\item[-]\small{Setting up our server infrastructure.}
\end{itemize}
\end{itemize}

\cvtag{Visual Basic 6}
\cvtag{MS SQL Server}
\cvtag{HTML}
\cvtag{JavaScript}



\medskip\medskip\cvevent{Software Developer}{Broekhuis Solutions}{1999 -- Jan 2000}{Rotterdam, The Netherlands}

\begin{itemize}
	\item\small{Part time work while doing my Master degree.}
	\item\small{Worked on Microsoft Access projects backed by a SQL Server database.}
	\item\small{Database optimization was a large part of this work.}
\end{itemize}

\cvtag{Microsoft Access}
\cvtag{MS SQL Server}









\newpage

\begin{fullwidth}

\cvsection[]{Achievements}
To give an impression of my work, below is a more detailed description of some past projects I’m particularly pleased with.

\bigskip

\cvsubsection{CapeVM}

During my PhD I developed a Java virtual machine for resource-constrained embedded CPUs. These typically have only a few KB of RAM, and in the order of tens of KBs of flash program memory. Their extremely low power consumption allows them to run for months or years on a single charge, if programmed correctly.

\medskip\medskip

Many proof-of-concept VMs have been developed for these devices, but, as interpreters, they are all one to two orders of magnitude slower than native code. This significantly increases energy consumption and thus negates the main advantage of these CPUs.

\medskip\medskip

My research showed that, even with so little resources, it is possible to translate Java bytecode to native code at load time, on the device. With a number of carefully designed optimizations, this leads to performance several times faster than existing VMs, and within half that of optimized native C.

\medskip\medskip

Besides platform independence, the resulting VM can also provide a safe execution environment, at a performance cost that is on-par with or lower than existing native code solutions.

\medskip\medskip

The paper describing the final result was accepted to SenSys 2018, the top conference in the field with a 16\% acceptance rate.

\bigskip\bigskip

\cvsubsection{Health insurance claims processing}{}{}{}
After working for a health insurance company for almost two years, I became the lead engineer for their new claims processing system, which at the time was done by a number of different systems. This project was part of the plan to migrate all legacy systems to .Net.

\medskip\medskip

I lead a team of 3-5 engineers to implement this system, designing many of the core parts of the system myself. The first version, initially processing a single (but complex) type of health care, was delivered in just 2,5 years. It has since been extended to replace all other claims processing systems, and over the past 13 years, has proven to be very maintainable, despite being the company’s largest and functionally most complex systems.

\medskip\medskip

When we started this project, the company had only recently started using .Net and much of the infrastructure needed to build a system like this was missing. As a result, we produced several spin-off projects that were subsequently adopted as standards by the rest of the company, many of which are still in use today:

\medskip\medskip

\begin{itemize}
	\item\small Generic background processing engine. To automatically process incoming claims and several related processes, we developed a uniform way to define these processes, handle incoming tasks, set priorities, create and configure worker threads, monitor their performance, handle errors, etc. This engine has since been generalized and made the standard to implement background processing for all other projects within the company.
	\item\small A standard set of code generation templates. These generate the database and data access layer, which enforced a level of consistency across projects. This also made it easy to generate other features from the entity definitions, such as views that hide privacy sensitive information when developers need to access production data, or an in-memory database which greatly sped-up unit testing.
	\item\small A simple workflow system to keep track of tasks requiring a user's attention. What pleased me about this spin-off in particular, is how the company was about to start down a traditional waterfall process asking users to come up with the requirements for the new workflow system. My gut feeling said that this would inevitably lead to a very complex solution, while for us, a simple to-do list would be sufficient. This intuition proved to be correct, and we implemented a simple approach with just a Task and TaskType table, which has since been used by all subsequent .Net projects with only minor extensions.
\end{itemize}

\bigskip\bigskip

\cvsubsection{Efficient code distribution in wireless sensor networks}{}{}{}
While doing my research on wireless sensor networks, reprogramming them by connecting a cable to each device quickly became tedious, so I developed a way to reprogram our nodes remotely.

\medskip\medskip

I developed a simple diff language that could update the code using the binary image already present on the device, using a fraction of the bandwidth that would be needed to transmit a complete new image.

\medskip\medskip

Similar to my work on CapeVM, a few binary optimizations were important. The key realization was that, after a small code change, most of the resulting binary code is identical, except for a shift in addresses. By including a short list of address ranges and offsets, the receiving node could path all memory accesses automatically, significantly reducing the size of the patch script.

\medskip\medskip

To date this is my most cited research paper.

\bigskip\bigskip

\cvsubsection{Automated deploys}{}{}{}
The company already had an automated deployment system running for a number of years which deployed 200+ modules to a range of target environments. In this system each module provided PowerShell scripts to deploy their code.

\medskip\medskip

It quickly became clear to me that this approach would become unmaintainable over time as the deploy scripts, often copy-pasted from one project to another, were starting to diverge and often conflicted.

\medskip\medskip

I proposed switching to an approach where teams specify what to deploy in an xml file stored in a central repository, and the deployment system decides how to deploy this using PowerShell scripts parameterized by this xml.

\medskip\medskip

The key insight was that since the deployment system is completely under our own control, it can be very limited in the functionality it offers since if needed, new functionality can be easily added. Ideally though, there should a single way to deploy something with as little variations as possible.

\medskip\medskip

This approach had several advantages
\begin{itemize}
\item\small It relieved teams from having to write scripts code to deploy their application.
\item\small It led to far greater consistency in how systems are deployed.
\item\small Deployed items are named consistently, making them easier to find and identify.
\item\small The central repository of xmls provides a global view of what is being deployed.
\item\small By modifying the scripts, we can change how things are deployed globally, for all modules at the same time. For instance, this has been used to switch to encrypted database connections without requiring any action from other teams.
\end{itemize}

\bigskip\bigskip

\cvsubsection{Service call tracing}{}{}{}
While working on designing an automated system to do integration testing of the company’s complicate application landscape, it became clear there was no insight into what service calls were being made between all the different systems. This meant that we did not know what scenarios should be tested to achieve good test coverage.

\medskip\medskip

At the time the company was almost exclusively using Windows Communication Foundation (WCF). To get an objective measurement of which systems were communicating, I developed a tracing tool that settles into the WCF stack and log all calls sent to, or originating from a server. By deploying this on both the production and testing environment and comparing the data, we were able to determine the test coverage of our integration tests.

\medskip\medskip

The insights obtained from this tracing also proved extremely valuable to monitor performance and analyze production issues, and we later extended it by adding tracing of several other events such web API calls, service bus messages, and the processing engine described earlier, and by adding correlation ids which allows us to follow a flow over multiple systems.

\medskip\medskip

Although more limited in scope, it often provided insights that could not have been obtained from far more expensive solutions like Dynatrace.





\newpage

\cvsection[]{Publications}

\cvsubsection{Journal Papers}

\cvpublication{Niels Reijers and Chi-Sheng Shih}{Improved Ahead-of-Time Compilation of Stack-Based JVM Bytecode on Resource-Constrained Devices}{ACM Transactions on Sensor Networks}{Aug 2019}

\medskip\medskip

\cvpublication{Niels Reijers, J. Ellul and Chi-Sheng Shih}{Making sensor node virtual machines work for real-world applications}{Embedded Systems Letters}{May 2018}

\medskip\medskip

\cvpublication{Koen Langendoen and Niels Reijers}{Distributed Localization in Wireless Sensor Networks: A Quantitative Comparison}{Computer Networks (Elsevier), special issue on Wireless Sensor Networks}{Aug 2003}

\bigskip\bigskip

\cvsubsection{Conference Papers}

\cvpublication{Niels Reijers and Chi-Sheng Shih}{CapeVM: A Safe and Fast Virtual Machine for Resource-Constrained Internet-of-Things Devices}{Proceedings of the 16th ACM Conference on Embedded Networked Sensor Systems (SenSys)}{Nov 2018}

\medskip\medskip

\cvpublication{Niels Reijers and Chi-Sheng Shih}{Ahead-of-Time Compilation of Stack-Based JVM Bytecode on Resource-Constrained Devices}{Proceedings of the International Conference on Embedded Wireless Systems and Networks (EWSN)}{Feb 2017}

\medskip\medskip

\cvpublication{Niels Reijers, Kwei-Jay Lin, Yu-Chung Wang, Chi-Sheng Shih, and Jane Y. Hsu}{Design of an Intelligent Middleware for Flexible Sensor Configuration in M2M Systems}{Proceedings of the 2nd International Conference on Sensor Networks (SENSORNETS)}{Feb 2013}

\medskip\medskip

\cvpublication{Kwei-Jay Lin, Niels Reijers, Yu-Chung Wang, Chi-Sheng Shih, and Jane Y. Hsu}{Building Smart M2M Applications Using the WuKong Profile Framework}{IEEE International Conference on Green Computing and Communications and IEEE Internet of Things and IEEE Cyber, Physical and Social Computing}{Aug 2013}

\medskip\medskip

\cvpublication{Niels Reijers, Yu Chung Wang, Chi-Sheng Shih, Jane Y. Hsu and Kwei-Jay Lin}{Building Intelligent Middleware for Large Scale CPS Systems}{Proceedings of the 2011 IEEE International Conference on Service-Oriented Computing and Applications (SOCA)}{Dec 2011}

\medskip\medskip

\cvpublication{Niels Reijers, Gertjan Halkes and Koen Langendoen}{Link Layer Measurements in Sensor Networks}{First IEEE International Conference on Mobile Ad hoc and Sensor Systems (MASS)}{Oct 2004}

\medskip\medskip

\cvpublication{Niels Reijers and Koen Langendoen}{Efficient Code Distribution in Wireless Sensor Networks}{Second ACM International Workshop on Wireless Sensor Networks and Applications (WSNA)}{Sep 2003}

\medskip\medskip

\cvpublication{Niels Reijers, Raymond Cunningham, René Meier, Barbara Hughes, Gregor Gärtner and Vinny Cahill}{Using group communication to support mobile augmented reality applications}{Proceedings Fifth IEEE International Symposium on Object-Oriented Real-Time Distributed Computing (ISORC)}{Apr 2002}



\bigskip\bigskip

\cvsubsection{Workshop}

\cvpublication{Chi-Sheng Shih and Niels Reijers}{Intelligent Middleware for Large Scale Cyber-Physical Systems}{half day workshop at 29th ACM Symposium On Applied Computing (SAC), Gyeongju, Korea}{Mar 2014}

\bigskip\bigskip

\cvsubsection{Book Chapter}

\cvpublication{Koen Langendoen and Niels Reijers}{Distributed Localization Algorithms}{in Embedded Systems Handbook, R. Zurawski (editor), CRC press}{Aug 2005}

\end{fullwidth}


\end{document}
